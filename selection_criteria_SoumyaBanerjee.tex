% Template for PLoS
% Version 1.0 January 2009
%
% To compile to pdf, run:
% latex plos.template
% bibtex plos.template
% latex plos.template
% latex plos.template
% dvipdf plos.template

\documentclass[10pt]{article}

% amsmath package, useful for mathematical formulas
\usepackage{amsmath}
% amssymb package, useful for mathematical symbols
\usepackage{amssymb}

% graphicx package, useful for including eps and pdf graphics
% include graphics with the command \includegraphics
\usepackage{graphicx}

\usepackage{url}
\usepackage{hyperref}

% cite package, to clean up citations in the main text. Do not remove.
\usepackage{cite}

\usepackage{color} 

% Use doublespacing - comment out for single spacing
\usepackage{setspace} 
\doublespacing

\usepackage{color}
\usepackage[usenames,dvipsnames]{xcolor}
\newcommand{\comment}[1]{{\color{red}\{#1\}}}


% Text layout
\topmargin 0.0cm
\oddsidemargin 0.5cm
\evensidemargin 0.5cm
\textwidth 16cm 
\textheight 21cm

% Bold the 'Figure #' in the caption and separate it with a period
% Captions will be left justified
\usepackage[labelfont=bf,labelsep=period,justification=raggedright]{caption}

% Use the PLoS provided bibtex style
\bibliographystyle{plos2009}

% Remove brackets from numbering in List of References
\makeatletter
\renewcommand{\@biblabel}[1]{\quad#1.}
\makeatother


% Leave date blank
\date{}

\pagestyle{myheadings}
%% ** EDIT HERE **


%% ** EDIT HERE **
%% PLEASE INCLUDE ALL MACROS BELOW

%% END MACROS SECTION

\begin{document}

% Title must be 150 characters or less
\begin{flushleft}
{\Large
\textbf{Responses to Selection Criteria}
}


% Insert Author names, affiliations and corresponding author email.

Soumya Banerjee$^{1}$
\\
\bf{1} University of Cambridge, Cambridge, United Kingdom
\\
%$\ast$ E-mail: Corresponding author XX@maths.ox.ac.uk
\end{flushleft}

% Please keep the abstract between 250 and 300 words
\section*{Selection Criteria}





\begin{enumerate}

\item TEST

\item TEST

\end{enumerate}



\begin{enumerate}

\item Experience of working with industry.

I have completed multiple projects in close collaboration with industry. These have been applied to a real world setting.

% Patient stratification frameworks implemented in clinical collaborations and published in \textit{NPJ Schizophrenia} \cite{Banerjee2021e}.

I have worked closely with industry to deliver results on how mathematical modelling can be used to prioritize patients for therapy \cite{Aschenbrenner2020}. 

Aschenbrenner D, Quaranta M, Banerjee S,  et al. (2020) Deconvolution of monocyte responses in inflammatory bowel disease reveals an IL-1 cytokine network that regulates IL-23 in genetic and acquired IL-10 resistance \textit{Gut} (impact factor = 19.8)

This project was conducted in collaboration with Eli Lilly pharmaceuticals.

I have also worked closely with supply chain management companies in Australia to build mathematical models of logistics \cite{Banerjee2016b}\cite{Garcia-Flores2016a}\cite{Tyshetskiy2016}. These have also included models for forecasting demand in ports in Sydney, Australia \cite{Tyshetskiy2016}.


\item A proven research track record of internationally high quality in Computer Science;

My research is in computational medicine, machine learning and data science applied to healthcare and modelling of biological systems. I build computer models to investigate problems related to human health.

I have over 10 years experience in computational biology, health informatics. I also focus on patient-centric and user-centric research in healthcare AI. 

I conduct high quality research in this field. 


I have multiple first author publications in journals like the Royal Society Interface and Nature Partner Journal Schizophrenia. I have the ability to publish high quality papers in leading journals \cite{Banerjee2022b,Banerjee2021e,Banerjee2018d,Banerjee2017g,Banerjee2016,Liu2014a,Mallick2019,Graessl2017,Banerjee2020d,Aschenbrenner2020}. I have published more than 10 papers in journals and peer-reviewed conferences. Some of my publications are in journals like \textit{Nature Communications} (impact factor = 11.8), \textit{Gut} (impact factor = 19.8) and \textit{Angewandte Chemie} (impact factor = 13.7). 


% I have experience in machine learning \cite{Banerjee2021e} and Bayesian methods \cite{Banerjee2017g}. 

Some of my most relevant publications are listed below:

Patient and public involvement to build trust in artificial intelligence: a framework, tools and case studies, Soumya Banerjee, Phil Alsop, Linda Jones, Rudolf Cardinal, \textit{Patterns} 3(6):100506, 2022

A class-contrastive human-interpretable machine learning approach to predict mortality in severe mental illness, S. Banerjee, P. Lio, P. Jones, R. Cardinal, \textit{Nature Partner Journal Schizophrenia}, 7, 60, 2021

Hydroxychloroquine: balancing the needs of LMICs during the COVID-19 pandemic, Soumya Banerjee, \textit{Lancet Rheumatology}, 2(7):385-386, 2020

Aschenbrenner D, Quaranta M, Banerjee S,  et al. (2020) Deconvolution of monocyte responses in inflammatory bowel disease reveals an IL-1 cytokine network that regulates IL-23 in genetic and acquired IL-10 resistance, \textit{Gut}, 2020 (impact factor = 19.8, British Medical Journal publishing group)

Influence of correlated antigen presentation on T cell negative selection in the thymus, Soumya Banerjee, SJ Chapman, \textit{Journal of the Royal Society Interface}, 15(148), 20180311, 2018

Modelling the effects of phylogeny and body size on within-host pathogen
replication and immune response, S. Banerjee, A. Perelson, M. Moses, \textit{Journal of the Royal Society Interface}, 14(136), 20170479, 2017 

Estimating biologically relevant parameters under uncertainty for experimental within-host murine West Nile virus infection, S. Banerjee, J. Guedj, R. Ribeiro, M. Moses and A. Perelson, \textit{Journal of the Royal Society Interface}, 13(117), 20160130, 2016

An excitable Rho GTPase signaling network generates dynamic subcellular
contraction patterns, M. Graessl, J. Koch, A. Calderon, S. Banerjee, T. Mazel, N.
Schulze, J. Jungkurth, A. Koseska, L. Dehmelt, P. Nalbant, \textit{Journal of Cell Biology}, 216(12), 4271-4285, 2017 (impact Factor = 9.7)

A bioorthogonal small-molecule switch system for controlling protein function in
cells, P. Liu, A. Calderon, G. Konstantinidis, J. Hou, S. Voss, X. Chen, F. Li, S. Banerjee, J. et al., \textit{Angewandte Chemie}, 53(38), 10049-10055, 2014 (impact factor = 13.7)

Predictive metabolomic profiling of microbial communities using amplicon or metagenomic sequences, H. Mallick, E. Franzosa, L. McIver, S. Banerjee, A. Sirota-Madi, A. Kostic, C. Clish, H. Vlamakis, R. Xavier, C. Huttenhower, \textit{Nature Communications}, 10(1):3136, 2019 (impact factor = 12.1)

Optogenetic tuning reveals Rho amplification-dependent dynamics of a cell contraction signal
network, D. Kamps, J. Koch, V. Juma, E. Campillo, M. Graessl, S. Banerjee, T. Mazel, X. Chen, Y.
Wu, S. Portet, A. Madzvamuse, P. Nalbant, L. Dehmelt, \textit{Cell Reports}, 33(9):108467, 2020 (Cell
Press publishing group, Impact Factor = 8.1)



\item A research programme that enhances and is coherent with the research profile of the Department
of Computer Science;

My research programme in computational biology, health informatics and patient-centric research in healthcare AI will complement existing research in the Department of Computer Science.

\item Ability to attract research funding and develop an independent programme of research;

I have the ability to attract research funding. I can develop an independent programme of research. I have recently been shortlisted for a pilot grant from the AI@CAM initiative (University of Cambridge). I will seek funding from institutions like the UKRI, EPSRC, BBSRC and the MRC.

\item Experience teaching Computer Science (or closely related subjects);

I have experience teaching a variety of subjects in Computer Science.

I have taught and am teaching the following subjects in the Computer Science department at the University of Cambridge:

\begin{enumerate}
	\item Data visualization and advanced data science
	\item Introduction to machine learning
	\item Reproducible analysis in data science
	\item Unconventional approaches to Artificial Intelligence
\end{enumerate}

I am very passionate about teaching. 
I have also co-designed teaching resources with educators to make my teaching broadly
accessible to classrooms \cite{Banerjee2019a}.

I also make my teaching resources freely available for use by others \cite{Banerjee2019a}.

A sample of my teaching material is available online 

https://osf.io/25gnz/

https://sites.google.com/site/neelsoumya/teaching


I am very passionate about using educational tools for outreach. I regularly review teaching
material for a non-profit organization called SIMIODE. My contributions as a reviewer of peer-reviewed
course material have been acknowledged.

\item The ability and willingness to teach effectively, both at undergraduate and graduate level, a wide
range of Computer Science topics;

I am able and willing to teach effectively at both undergraduate and graduate level.

I have taught and am teaching the following subjects in the Computer Science department at the University of Cambridge:

\begin{enumerate}
	\item Data visualization and advanced data science
	\item Introduction to machine learning
	\item Reproducible analysis in data science
	\item Unconventional approaches to Artificial Intelligence
\end{enumerate}


I am also applying for fellowship of the Higher Education Academy (HEA).


\item Ability to supervise graduate students;

I am able to supervise and am supervising graduate students. I have also supervised 5 MPhil students and am currently supervising 6 MPhil students.

\item Excellent interpersonal skills necessary for undertaking tutorial teaching and the pastoral care of
students;


I have a high-level of inter-personal and communication skills.
I have excellent written and verbal communication skills as evidenced by my publications and 5 invited talks and 5 contributed talks.

I have the ability to teach and relate to diverse group of students. I can conceive and implement lessons and activities that can cater to the individual learning needs of the students.

I have also supervised 5 MPhil students and am currently supervising 6 MPhil students. I have the skills necessary for tutorial teaching and the pastoral care of students.



\item Ability and willingness to undertake the full range of administrative duties both within the
department and the College.

I am able and willing to undertake all administrative duties in the Department and the College. In my current role, I also perform a wide range of administrative duties.


\item Experience in conducting, documenting and communicating research required.

I have over 10 years experience in modelling and computational mathematics in different fields of computational biology ranging from immunology and virology to cell biology and metagenomics. I am committed to building a research program that computational biology to different biological systems of clinical and public health significance.
 I have applied biostatistical techniques to three different disease systems: West Nile virus infection, cancer biology and inflammatory bowel disease. I have published more than 10 papers in journals and peer-reviewed conferences. One of my publications is in the journal Angewandte Chemie (impact factor = 13.7).


\item Ability to work unsupervised and under own initiative is required.

I am able to work unsupervised and on my own initiative.


\item Experience in teaching and/or supporting junior staff is desirable.

I have some experience teaching and mentoring junior staff and students.


\item At least one first/author research article published, in press or accepted for publication in a major peer-reviewed journal is required.


I have multiple first author publications in journals like the Royal Society Interface. I have the ability to publish high quality papers in leading journals \cite{Banerjee2018d,Banerjee2017g,Banerjee2016,Liu2014a,Mallick2019,Graessl2017,Banerjee2020d,Aschenbrenner2020}. I have published more than 10 papers in journals and peer-reviewed conferences. Some of my publications are in journals like \textit{Nature Communications} (impact factor = 11.8), \textit{Gut} (impact factor = 19.8) and \textit{Angewandte Chemie} (impact factor = 13.7). 


I have experience in machine learning \cite{Banerjee2021e} and Bayesian methods \cite{Banerjee2017g}. 

Some of my most relevant publications are listed below:


Some of my most relevant publications are listed below:

Patient and public involvement to build trust in artificial intelligence: a framework, tools and case studies, Soumya Banerjee, Phil Alsop, Linda Jones, Rudolf Cardinal, \textit{Patterns} 3(6):100506, 2022

A class-contrastive human-interpretable machine learning approach to predict mortality in severe mental illness, S. Banerjee, P. Lio, P. Jones, R. Cardinal, \textit{Nature Partner Journal Schizophrenia}, 7, 60, 2021

Hydroxychloroquine: balancing the needs of LMICs during the COVID-19 pandemic, Soumya Banerjee, \textit{Lancet Rheumatology}, 2(7):385-386, 2020

Aschenbrenner D, Quaranta M, Banerjee S,  et al. (2020) Deconvolution of monocyte responses in inflammatory bowel disease reveals an IL-1 cytokine network that regulates IL-23 in genetic and acquired IL-10 resistance, \textit{Gut}, 2020 (impact factor = 19.8, British Medical Journal publishing group)

Influence of correlated antigen presentation on T cell negative selection in the thymus, Soumya Banerjee, SJ Chapman, \textit{Journal of the Royal Society Interface}, 15(148), 20180311, 2018

Modelling the effects of phylogeny and body size on within-host pathogen
replication and immune response, S. Banerjee, A. Perelson, M. Moses, \textit{Journal of the Royal Society Interface}, 14(136), 20170479, 2017 

Estimating biologically relevant parameters under uncertainty for experimental within-host murine West Nile virus infection, S. Banerjee, J. Guedj, R. Ribeiro, M. Moses and A. Perelson, \textit{Journal of the Royal Society Interface}, 13(117), 20160130, 2016

An excitable Rho GTPase signaling network generates dynamic subcellular
contraction patterns, M. Graessl, J. Koch, A. Calderon, S. Banerjee, T. Mazel, N.
Schulze, J. Jungkurth, A. Koseska, L. Dehmelt, P. Nalbant, \textit{Journal of Cell Biology}, 216(12), 4271-4285, 2017 (impact Factor = 9.7)

A bioorthogonal small-molecule switch system for controlling protein function in
cells, P. Liu, A. Calderon, G. Konstantinidis, J. Hou, S. Voss, X. Chen, F. Li, S. Banerjee, J. et al., \textit{Angewandte Chemie}, 53(38), 10049-10055, 2014 (impact factor = 13.7)

Predictive metabolomic profiling of microbial communities using amplicon or metagenomic sequences, H. Mallick, E. Franzosa, L. McIver, S. Banerjee, A. Sirota-Madi, A. Kostic, C. Clish, H. Vlamakis, R. Xavier, C. Huttenhower, \textit{Nature Communications}, 10(1):3136, 2019 (impact factor = 12.1)

Optogenetic tuning reveals Rho amplification-dependent dynamics of a cell contraction signal
network, D. Kamps, J. Koch, V. Juma, E. Campillo, M. Graessl, S. Banerjee, T. Mazel, X. Chen, Y.
Wu, S. Portet, A. Madzvamuse, P. Nalbant, L. Dehmelt, \textit{Cell Reports}, 33(9):108467, 2020 (Cell
Press publishing group, Impact Factor = 8.1)


The early impact of COVID-19 on mental health and community physical health services and their patients’ mortality in Cambridgeshire and Peterborough, UK, S. Chen, P. Jones, B.
Underwood, A. Moore, E. Bullmore, S. Banerjee, et al., \textit{Journal of Psychiatric Research}, 131, 244-254, 2020 (Impact factor = 4.4)

Simulating a community mental health service during the COVID-19 pandemic: effects of clinician-clinician encounters, clinician-patient-family encounters, symptom-triggered protective behaviour, and household clustering, R. Cardinal, C. Meiser, D. Christmas, A. Price, C. Denman, B. Underwood, S. Chen, S. Banerjee \textit{Frontiers in Psychiatry}, 12, 196, 2021 (Impact Factor = 3.5) 

A relevant paper in epidemiology that is under review is listed below:

A human-interpretable machine learning approach to predict mortality in severe mental illness, S. Banerjee, P. Lio, P. Jones, R. Cardinal, \textit{medRxiv}, 4, 2021 (under review)


%A relevant paper in the field of Computational Medicine (in press in the \textit{Gut) is also listed below \cite{Aschenbrenner2019}
%
%Aschenbrenner D, Quaranta M, Banerjee S, Ilott N, Jansen J, et al. (2019) Systems-level analysis of monocyte responses in inflammatory bowel disease identifies IL-10 and IL-1 cytokine networks that regulate IL-23. bioRxiv:719492 (under review)


\item 
Mandatory skills are;
- Strong mathematical and computational background, particularly stochastic processes
- Programming skills and experience writing simulations
- Critical thinking and the ability to rigorously test predictions in multiple independent ways


I have experience in stochastic processes. I also have experience in multi-scale simulations using genomic and single-cell sequencing data.
These multi-scale simulations were coupled to large amounts of genomic data and were validated experimentally.


\item Additionally, substantial research experience is valued in the following;
- Experience in working with genomic data sets
- Experience with python and latex
- Experience mentoring others


I have experience in multi-scale simulations using genomic and single-cell sequencing data. I also have extensive experience in working with genomic datasets.

I also have a lot of experience programming in python (especially building stochastic simulators).

I am also very proficient in latex.

I have mentored PhD students.



\item Ability to manage own workload in an independent way and to deal appropriately with competing priorities.

In my PhD and post-doc, I have initiated two research projects. I have the ability to work independently and deal with competing priorities.



\item A PhD (or equivalent) in a relevant subject such as mathematical, physical or biological sciences

I have completed my PhD in Computer Science from the University of New Mexico, USA.


\item Fellowship of the Higher Education Academy (A/C/I) and professional body membership such as Member of the British Computer Society

I do not have these affiliations. However I am an active member and part of the organizing committee of Artificial Immune Systems.


\item Experience of teaching in Higher Education

I have experience of teaching in Higher Education especially at the Undergraduate and Postgraduate level.


\item Communication and interpersonal skills required.

I have a high-level of inter-personal and communication skills.
I have excellent written and verbal communication skills as evidenced by my publications and 5 invited talks and 5 contributed talks.


\item Ability to teach effectively and communicate with audiences and staff working with different technical and education backgrounds required.

I have a high-level of inter-personal and communication skills.
 I also work extensively with my collaborators in the experimental sciences. 
 I enjoy communicating mathematical ideas, models and statistical results to non-statisticians including biologists and clinicians.
I have experience working with domain experts from different fields and communicating results to diverse stakeholders.
I have extensive experience working with experimentalists and have wet-lab experience. I am also very passionate about making computational and statistical tools accessible to biologists. To this end I have conducted lectures that demystify computational techniques for experimentalists. 


\item Broad experience in curriculum development for health and/or information technologies

I have experience developing curricula for health and information technologies.


\item Demonstrate evidence of the development of innovative teaching techniques that are designed to create
interest, understanding and enthusiasm among students; encouraging them to engage in the highest levels of scholarship and achievement.

I am able to develop innovative teaching techniques that will engage students in the classrooms and lead them to the highest levels of academic and professional achievement.


\item The ability to relate to a diverse cohort of students, including prospective, enrolled and graduated students,  with the capacity to conceive, initiate and run activities relevant to each of these different student groups.



I have the ability to teach and relate to diverse group of students. I can conceive and implement lessons and activities that can cater to the individual learning needs of the students.



\item A record of achievement in developing
curricula and resources that is research led and
led to improved student outcomes.

I design student projects that are currently relevant problems in industry. This makes students more employable after graduation.

I have a record of developing curricula that is driven by current industry-standard research that can lead to better employment and learning outcomes for students.



\item Ability to work productively and cordially as part of a team.

I am able to work effectively in teams. I worked in and managed teams while working in industry.


\item An ability to innovate, including the development of new modules

I am able to innovate in teaching in the classroom and contribute to development of new modules


I have the ability to produce high quality curricula and program materials.

I use technology to engage students. These include graphical user interfaces and interactive
environments. These tools complement my hands-on teaching style. I encourage my students to
play around with computer programs and user interfaces to build interactive prototypes to solve
problems in complex systems.



I am very passionate about teaching. 
I have also co-designed teaching resources with educators to make my teaching broadly
accessible to classrooms \cite{Banerjee2019a}.

I also make my teaching resources freely available for use by others \cite{Banerjee2019a}.

A sample of my teaching material is available online 

https://osf.io/25gnz/

https://sites.google.com/site/neelsoumya/teaching


I am very passionate about using educational tools for outreach. I regularly review teaching
material for a non-profit organization called SIMIODE. My contributions as a reviewer of peer-reviewed
course material have been acknowledged.
I have also co-designed teaching resources with educators to make my teaching broadly
accessible to classrooms \cite{Banerjee2019a}.

A sample of my teaching material is available online

https://www.simiode.org/resources/3206

https://osf.io/25gnz/



\item An excellent understanding of underlying theory and current, industrial practice in
software and/or data engineering

I have an excellent understanding of the theory and industrial practice in software engineering, data engineering and data science.


I have worked in industry (in the financial services sector) for Fortune 500 clients.

I also have experience engaging with industry and other diverse stakeholders. I interact with pharmaceutical companies and other industry partners.

I have knowledge of practical issues related to software and data science in industry.
I would like to teach students about practices that are relevant in industry. 

I would like to design curricula that is driven by current industry relevant problems and practices. This will lead to better employment and learning outcomes for students.


\item An aptitude for teaching, and the skills and judgement needed to teach advanced
concepts and techniques to working professionals with a range of experiences and
expectations.

I have an aptitude for teaching. I have the skills needed to teach advanced concepts and practical techniques to working professionals.

I have worked in industry and I would like to teach about practical skills that are relevant to professionals.




\item A genuine enthusiasm for teaching and learning, and the ability to work effectively as
part of a team of academic and administrative staff responsible for delivery

I am very passionate about teaching, especially skills that can be used in industry.
The skills are related to designing and implementing production systems in data science and software.

I have the ability to work in a team of staff responsible for delivery.



\item Experience of software, data, or security engineering in a production context

I have experience working in industry and have designed and implemented software systems running in production on critical financial data.

I also have experience engaging with industry and other diverse stakeholders. I have interacted with pharmaceutical companies and supply chain management companies. I would like to design student projects that are currently relevant to problems faced by industry. 







\item An ability to develop a portfolio of external income generation

I have the ability to generate external income by applying for grants from government agencies and industry.

\item A track record of securing research funding grant

I have the ability to raise research funds through grant applications. I have submitted grant applications as a co-principal investigator to GCRF (Global Challenges Research Fund).

I have the ability to independently plan and manage research projects. I have experience making research budgets.



\item Strong industry and community engagement experience. Proven ability to communicate effectively with academic staff, researchers, students, government and health industry professionals

I have experience engaging with industry and other diverse stakeholders. I interact with pharmaceutical companies and other industry partners.


\item A flexible approach to work and the ability to engage with colleagues and scientists from different backgrounds and fields are required.

I believe collaboration is key to progressing science. I collaborate and interact extensively with my colleagues. I work with my colleagues in biostatistics and mathematics to develop new statistical techniques to analyse biological data. I also work extensively with my collaborators in the experimental sciences, especially immunologists, virologists and cell biologists to apply mathematical techniques to their problems.


\item Time management skills required.

I am very adept at managing my time.


\item  Well organized with good time management skills 

I am very well organized with good time management skills and can effectively prioritise multiple tasks.

\item Post holder may be required to attend courses or relevant international meetings to enable continuous professional development and remain up to date with current developments in the field.

I am able to and willing to attend courses and relevant meetings to develop myself professionally.


\item Flexible approach to work associated with requirements for research involving interdisciplinary techniques is essential, which will require ability to adapt the working schedule to varying hours at a short notice (e.g. late work).

I am able to work in a flexible and adaptive manner to ensure that deliverables and deadlines are met.




\item Evidence of ability to work collaboratively across disciplines.


I believe collaboration is key to progressing science. I collaborate and interact extensively with my colleagues. I work with my colleagues in biostatistics and mathematics to develop new statistical techniques to analyse biological data. I also work extensively with my collaborators in the experimental sciences, especially immunologists, virologists and cell biologists to apply mathematical techniques to their problems.


\item Willingness to continually update knowledge in the specialist area and engage in continuous professional development.

I am willing to continuously update my knowledge in specialist areas and engage in continuous professional development.


\item Strong programming skills, preferably in Python or Java.

I have strong programming skills in Python, R and MATLAB.
I am ranked within the top 200 worldwide on MATLAB Central (an online repository for Matlab code contributed by users all over the world). Prior to graduate school, I was a software engineer working in the financial services sector for Fortune 500 clients. Because of this experience, I write well-documented and rigorously tested code.

Examples of my code and open source projects are available at

https://bitbucket.org/neelsoumya/public\_open\_source\_datascience

https://sites.google.com/site/neelsoumya/software

https://github.com/neelsoumya



\end{enumerate}


%\section*{Desirable Criteria}
%
%
%
%
%
%\begin{enumerate}
%\item Interest in intradisciplinary and interdisciplinary collaborations in support of existing departmental strengths
%\item University teaching in
%mathematics / statistics
%or related subjects to
%undergraduate or
%postgraduate students.
%\end{enumerate}

%
%\begin{equation}
%Pr (\text{escape or survive}) = (1 - d/m )^{ (1000 * n)} 
%\label{prob_escape_simple}
%\end{equation}
%\label{prob_escape_complex_part2}


%\subsection*{Dynamical Model}

%randomly generated TCR sequence. The results are shown in
%Fig. \ref{figure_analysis_degeneracy_SYNTHETIC}.

%\begin{figure}[!ht]
%\begin{center}
%\includegraphics[width=6in]{figures/analysis_degeneracy_SYNTHETIC.eps}
%\end{center}
%\caption{
%{\bf Distribution of number of degenerate interactions that 1000 randomly generated TCR sequences have with all self-peptides.  }  
%}
%\label{figure_analysis_degeneracy_SYNTHETIC}
%\end{figure}


%from agent_based_model_mesa/results/analysis_degeneracy_SYNTHETIC.eps
% generated from analysis_degeneracy_SYNTHETIC.R
% generated from python3 ImmuneModel_nomovement_SYNTHETIC_CHECKDEGENERACY_GIANTCELL.py


%\section*{References}
% The bibtex filename

% BIB FILE HERE
%\bibliography{Tcell_project}
\bibliography{all_my_papers_ref,cam_project}


%\begin{figure}[!ht]
%\begin{center}
%\includegraphics[width=6in]{figures/percentage_TCR_escape_vs_interactions_SYNTHETICDATA_1e4tcell.eps}
%\end{center}
%\caption{
%{\bf Fraction of thymocytes that escape from stochastic simulator as a function of the number of interactions with mTECs.  }  
%}
%\label{fraction_escape_figure}
%\end{figure}
%
%%from agent_based_model_mesa/results/percentage_TCR_escape_vs_interactions_SYNTHETICDATA_1e4tcell.eps
%%generated from python3 iterate_ImmuneModel_nomovement_SYNTHETIC.py

%\begin{figure}[!ht]
%\begin{center}
%\includegraphics[width=6in]{figures/percentage_TCR_escape_vs_interactions_iterate_ImmuneModel_nomovement_CHECKAUTO_AIREKO_SYNTHETIC_40pPYRO.png}
%\end{center}
%\caption{
%{\bf Fraction of thymocytes that escape from stochastic simulator.  }  
%}	
%\label{fraction_escape_figure_temp}
%\end{figure}

% Code file is XX %tested some of these results against synthetic AIRE KO (40%)
% and report statistics XX


%\begin{figure}[!ht]
%\begin{center}
%\includegraphics[width=6in]{figures/percentage_TCR_escape_vs_interactions_SYNTHETICAIREWT_FIXEDTCR_LOWDEGENERACYTCR.png}
%\end{center}
%\caption{
%{\bf Equation \ref{replicon_eqn4} fit to data on luciferase activity in mice using a WNV replicon (data from \cite{Lim2011}).}  
%}
%\label{replicon_figure}
%\end{figure}

% Code file is XX
% and report statistics XX




\end{document}


